\chapter*{Avant-propos}

La mise en page de ce document est calquée sur le modèle de la revue scientifique \href{https://peerj.com/}{PeerJ}. Je tiens à remercier sincèrement Pete Binfield, le co-fondateur de la revue, de nous avoir donné la permission d'utiliser son modèle. J'encourage les étudiants à visiter le site de la revue. Tous les articles scientifiques qu'elle contient sont en libre accès (gratuit pour les lecteurs).\\

Le contenu des notes de cours a évolué à partir d'un document de départ rédigé par Alain Cloutier. Depuis 2007, Alexis Achim et Alain Cloutier sont co-responsables du cours et ils ont travaillé ensemble à la mise à jour de certaines parties. La partie dur la physiologie de l'arbre du chapitre 2 est issue des notes de cours de David Pothier. Nous tenons à le remercier d'avoir partagé ses notes avec nous.\\

Plusieurs des illustrations ont été développées par Julie Ferland. Nous souhaitons aussi à souligner la contribution extraordinaire de Jean-Romain Roussel qui a réalisé la migration entre les formats MSWord et \LaTeX en plus de travailler à la production de certaines des illustrations. Si le format des notes de cours peut faciliter votre apprentissage, c'est grâce à lui.\\

Toutes les images et le texte de ce document sont placés sous licence \href{https://creativecommons.org/licenses/by-nc-sa/4.0/}{Creative Commons CC BY-NC-SA 4.0}
, ce qui implique que vous pouvez les réutiliser, modifier et partager dans un but non commercial et sous la même licence, et sous réserve de citer les auteurs originaux.\\

La citation de cet ouvrage devrait apparaître ainsi: Achim, A., Cloutier, A. 2018, GBO-4000 / GBO-7000 Anatomie et structure du bois -- Notes de cours, Automne 2018, Université Laval, 104 p.\\

 
 Bonne lecture et bonne session.\\
 
 \begin{flushright}
 	Alain Cloutier et Alexis Achim, professeurs
 \end{flushright}

\begin{figure}[ht]
	\centering
	\includegraphics[height=2cm]{img/CC_licence}
\end{figure}

