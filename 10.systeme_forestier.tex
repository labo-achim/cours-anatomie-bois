\chapter{Le système forestier québécois}

\begin{abstract}
Résumé
\end{abstract}

\minitoc

\section{Introduction}

Ce chapitre vise à décrire les bases du système forestier québécois et son impact sur les caractéristiques des approvisionnements aux usines de transformation. Une emphase particulière est portée au domaine de la forêt publique, puisqu'elle représente près de 90\% de la superficie productive de la province. La mise en œuvre d'un nouveau régime forestier en 2013 fait en sorte qu'il existe une certaine part d'incertitude quant à la direction que prendra la production forestière sur terres publiques. On peut cependant déjà entrevoir certains impacts des changements proposés sur les caractéristiques des approvisionnements. Avant d'entrer dans les détails de ces changements, il importe de rappeler l'évolution passée de notre sytème forestier.

\section{Grandes étapes du régime forestier dans les forêts publiques du Québec}

\begin{description}
\item[1822] L'état impose les permis de coupe et les redevances forestières pour le bois récolté sur les terres de la couronne. 
\item[1934] Le gouvernement du Québec instaure le régime des concessions forestières. Les compagnies se voient attribuer de vastes territoires dont elles contrôlent l'accès.
\item[1986] L'Assemblée nationale révoque les concessions forestières et instaure la loi sur les forêts. Ce ne sont plus les territoires qui sont alloués aux industries forestières, mais des volumes de bois. Par le biais des contrats d'aménagement et d'approvisionnement forestier (CAAF) une industrie se voit attribuer un volume annuel de bois pour une période de 25 ans (avec une révision quinquennale). Ce régime est en grande partie celui qui est toujours en place aujourd'hui (bien qu'il fût révisé en 2001). 

\end{description}

Par cette loi, le bénéficiaire de CAAF doit entre autres :

\begin{itemize}
\item respecter la possibilité annuelle de coupe à rendement soutenu;
\item réaliser les travaux sylvicoles nécessaires pour assurer la régénération d'un peuplement équivalent à celui qui a été récolté;
\item s'assurer que les travaux de récolte respectent la faune et les habitats fauniques, les cours d'eau et les rives, les sols, les besoins des autres utilisateurs;
\item payer des redevances qui reflètent la valeur des bois récoltés.

\end{itemize}


\begin{description}
\item[Unités d'aménagement forestier (UAF)] Superficie délimitant les limites territoriales de l'application d'un CAAF. Souvent on retrouve plusieurs bénéficiaires de CAAF sur la même UAF. Les bénéficiaires nomment un mandataire qui sera responsable d'interagir avec le MRNF au nom de tous les bénéficiaires. Ils nomment aussi des mandataires d'opération qui seront responsables de mener la logistique des opérations forestières sur le territoire. Il en va donc de la responsabilité des industriels de s'entendre sur le fonctionnement des activités de planification et d'opérations forestières dans l'aire commune (ententes cadres).

\item[Plan général d'aménagement forestier (PGAF)] Un plan qui, pour une UAF donnée, présente les stratégies d'aménagement forestier qui permettront la régénération des peuplements forestiers et qui permettront de respecter les autres bénéfices apportés par la forêt. Le plan général doit être rendu disponible au public pour consultation avant son adoption. Il contient les calculs de possibilité forestière (effectués sous la responsabilité du MRNF). À l'heure actuelle, ce plan n'offre pas de représentation spatiale des stratégies à appliquer.

\item[Plan quinquennal d'aménagement forestier (PQAF)] Décrit le programme des travaux sylvicoles à effectuer dans une UAF pour une période de cinq ans. Ce plan offre une description spatiale des travaux qui sont prévus. Il doit aussi être soumis à un processus de consultation publique.

\item[Plan annuel d'interventions forestières (PAIF)] Décrit le programme des travaux sylvicoles à effectuer dans une UAF sur une période d'un an. Ce plan n'est pas soumis à un processus de consultation publique, mais il doit se limiter aux balises fixées dans le plan quinquennal (sans quoi on doit faire une demande de modification et retourner en consultation publique).

\marginpar{Le RNI est sur le point d'être remplacé par le Règlement sur l'aménagement durable des forêts (RADF).}
\item[Le Règlement sur les normes d'intervention dans les forêts du domaine de l'État (RNI)] Définit les façons d'effectuer les travaux forestiers pour assurer le renouvellement de la forêt et la protection de l'eau, de la faune, de la végétation et du sol. Il s'agit d'un cahier de normes à respecter lors de la réalisation de travaux sylvicoles. Il définit, par exemple, la largeur des bandes riveraines à laisser intactes et la dimension maximale des coupes.

\item[Le manuel d'aménagement forestier] \marginpar{Le manuel est sur le point d'être remplacé par les guides sylvicoles, qui laisseront plus de marge de manœuvre aux ingénieurs forestiers.} Manuel élaboré pour encadrer l'aménagement forestier sur le territoire faisant l'objet de CAAF. On y décrit entre autres les méthodes et les hypothèses de calcul utilisées pour déterminer la possibilité annuelle de coupe à rendement soutenu.


\end{description}

\section{Vers un nouveau régime forestier}

En avril 2010, l'Assemblée nationale a sanctionné une nouvelle loi sur l'aménagement durable du territoire forestier. Cette loi définit le nouveau régime forestier qui sera mis en place en 2013.\\

Dorénavant, le Ministère sera responsable de l'aménagement durable des forêts du domaine de l'État et de leur gestion. En pratique, pour les périodes suivant 2013, il réalisera :

\begin{itemize}
\item la planification forestière ;
\item les interventions en forêt, leur suivi et leur contrôle ;
\item le mesurage des bois.
\end{itemize}

Le Ministère demeure responsable de l'attribution des droits forestiers (allocations de volumes de bois). D'autres changements importants sont prévus à la loi :

\begin{itemize}
\item l'adoption d'une approche d'aménagement durable qui passe notamment par l'aménagement écosystémique;
\item la mise de côté du principe du rendement soutenu au profit d'une conception plus large de l'aménagement forestier. Cette conception inclut l'ensemble des fonctions de la forêt permettant d'assurer la pérennité et l'utilisation diversifiée du milieu forestier;
\item la création de tables locales de gestion intégrée des ressources et du territoire coordonnées par les commissions régionales des ressources naturelles et du territoire;
\item l'octroi de garanties d'approvisionnement aux usines actuellement bénéficiaires de contrats d'approvisionnement et d'aménagement forestier et, éventuellement, à d'autres usines de transformation primaire ou de fabrication des produits à valeur ajoutée;
\item l'instauration, au sein du ministère des Ressources naturelles et de la Faune, d'une unité administrative ( \og Bureau de mise en marché des bois  \fg) chargée d'effectuer la vente de bois des forêts du domaine de l'État sur un marché libre; 
\item la désignation d'aires présentant un intérêt particulier pour la production ligneuse intensive.
\end{itemize}

\section{La forêt privée}

La forêt privée ne représente que 11\% du territoire forestier productif de la province mais, comme elle est concentrée au sud où les conditions de croissance sont favorables, elle compte pour 20\% des approvisionnements aux usines.
\\

La plupart des forêts privées sont de petite taille (88\% sont en deçà de 50 hectares). Pourtant, plusieurs propriétaires sont actifs sur le plan de la production forestière. Comment expliquer ce dynamisme ?
\\

Tout propriétaire d'une superficie boisée de plus de 4 ha peut s'enregistrer en tant que producteur forestier. Il a alors accès à plusieurs programmes d'aide financière :

\begin{itemize}
\item Le programme d'aide à la mise en valeur des forêts privées  (permet au propriétaire d'obtenir des subventions pour faire des travaux sylvicoles)
\item Le programme d'aide à l'aménagement des ravages de cerfs de Virginie 
\item Le programme de remboursement de taxes foncières 
\item Le programme de financement forestier
\end{itemize}

Les propriétaires forestiers sont regroupés dans la Fédération des producteurs de bois du Québec qui a pour mission de les représenter et de défendre leurs intérêts. La fédération comprend les différents syndicats et offices de producteurs de bois du Québec qui organisent et négocient la vente de bois aux industries transformation. Ainsi, même de faibles volumes de bois peuvent être mis en marché grâce à cet effort concerté.\\

Un propriétaire forestier pour se joindre à un organisme de gestion en commun (OGC). Il s'agit de regroupements de propriétaires qui mettent des ressources en commun pour être en mesure d'aménager leurs boisés. Pour plus d'information : http://www.resam.org/\\

Pour avoir accès aux différents programmes d'aide financière, un propriétaire forestier doit passer soit par un OGC ou par un conseiller forestier accrédité.
\\

Pour espérer obtenir des approvisionnements provenant des terres publiques, une compagnie forestière doit démontrer qu'elle écoule d'abord le bois disponible sur terres privées (principe de résidualité).

\section{La chaîne de production forestière}

La chaîne de production forestière comprend toutes les activités de production de matière ligneuse, de la régénération des semis au recyclage des produits, en passant par les soins culturaux, les opérations de récolte, la première et la deuxième transformation des produits.\\

Au Québec, près de 60 Mm\up{3} de bois ont été soumis à un processus de transformation primaire en 2007 (Tableau 10.1). Ce chiffre est bien supérieur à la possibilité forestière qui était d'environ 40 Mm\up{3} pour la même période. Il faut voir que le 60 M m3 inclut une grande part (20 Mm\up{3}) de copeaux et de résidus du sciage.

%Tableau 10.1 SEPM: Sapin, épinettes, pin gris, mélèze. Source : MRNF 2009

%Tableau 10.2. Source: MRNF 2009.

L'approvisionnement des usines de pâtes et papiers au Québec est dit  \og résiduel  \fg, c'est-à-dire qu'elles s'approvisionnent d'abord des résidus du sciage (Tableau 10.2). La grande majorité du bois rond est donc d'abord dirigé vers les scieries (Tableau 10.3). Pour le sylviculteur, ce système implique que l'objectif de production visera souvent la production de billes destinées à cette industrie.\\

Il faut comprendre également que plus de 8 Mm\up{3} provenaient de l'extérieur du Québec. La loi sur les forêts de 1986, sauf dans des circonstances particulières, obligeait les industries à transformer le bois provenant des terres publiques avant de l'exporter. Comme notre capacité de transformation est parfois plus grande que la capacité de la forêt à produire de la matière ligneuse, la province est une importatrice nette de bois rond (Figure 10.1). C'est notamment le cas dans l'industrie du sciage de bois feuillu.\\ 

La fibre qui ne se retrouve pas dans les produits du sciage (et autres produits solides) a de fortes chances de se retrouver dans les pâtes et papiers (Tableau 10.1). Cette industrie produit aussi des résidus qui sont le plus souvent réutilisés ou recyclés (Figure 10.2).

%Tableau 10.3 Source : MRNF 2009
%Figure 10.1  Source : MRNF, 2009
%Figure 10.2  Source : MRNF 2008

\section{Impact des changements du système forestier sur la quantité et la qualité des approvisionnements}

\subsection{Révision de la possibilité forestière}
La loi sur les forêts  de 1986 stipulait que les volumes récoltés sur terres publiques ne doivent pas affecter la capacité de la forêt de produire un rendement en bois soutenable à perpétuité. C'est ce qu'on appelle la possibilité annuelle de coupe à rendement soutenu. Suite à une recommandation de la Commission d'étude sur la gestion de la forêt publique québécoise (Commission Coulombe), la quantité de bois issue des terres publiques et attribuée à des usines de transformation (possibilité forestière) a été révisée à la baisse en 2008. La Commission remettait notamment en cause les méthodes et hypothèses de calcul. Suite à ses recommandations, la responsabilité du calcul de la possibilité forestière a été donnée à une entité indépendante du MRNFQ : le bureau du Forestier en Chef.\\
 
Le bureau du Forestier en Chef, tenant compte des recommandations de la Commission, a revu la possibilité annuelle de coupe à la baisse pour la plupart de nos essences commerciales (tableau 10.4). On peut se demander ce que l'abandon du principe de rendement soutenu amènera comme changement à la quantité de bois issu des terres publiques après 2013. L'avenir nous le dira…

%Tableau 10.4	Possibilité forestière annuelle par essence en milliers de m3. Source, bureau du forestier en chef (2007)
%
%1	Sapin, épinettes, pin gris, mélèze
%2	Érable à sucre, érable rouge

\subsection{Changements reliés à la mise en œuvre de l'aménagement écosystémique}

La définition de l'aménagement écosystémique retenue par le MRNF est la suivante : 

\begin{quotation}\it
L'aménagement écosystémique vise, par une approche écologique appliquée à l'aménagement forestier, à assurer le maintien de la biodiversité et de la viabilité de l'ensemble des écosystèmes forestiers tout en répondant à des besoins socio-économiques dans le respect des valeurs sociales liées au milieu forestier.
\end{quotation}

Le principal moyen utilisé pour atteindre les objectifs fixés consiste à tendre vers l'état  \og naturel  \fg de la forêt, c'est-à-dire son état en absence de perturbations anthropiques. Sa mise en œuvre passe donc par les étapes suivantes :

\begin{enumerate}
\item Obtenir un portrait de la composition et de la structure de la forêt naturelle. Il s'agit d'utiliser diverses sources historiques (relevés d'arpentage, inventaires forestiers des premières compagnies forestières, actes notariés, etc.) afin de reconstituer l'écosystème forestier préindustriel et, si possible, en estimer la variabilité dans le temps.
\item Établir des seuils d'altération acceptables. On reconnait que l'être humain aura un impact sur l'écosystème forestier. Cette étape consiste à estimer quel serait le seuil d'altération acceptable. Il en découle la formulation de cibles à atteindre pour les aménagistes forestiers.
\item Planification d'une séquence de travaux sylvicoles qui viseront à atteindre ces cibles.
\item Mise en œuvre sur le terrain et suivi.

\end{enumerate}

La mise en œuvre répondra à des enjeux régionaux fort variés, ce qui implique qu'il est difficile d'anticiper tous les changements qu'elle induira sur les approvisionnements des usines. On peut toutefois déjà entrevoir deux changements importants à l'échelle de la province.\\

Une augmentation de la disponibilité des feuillus intolérants (bouleau à papier, peuplier). Les diverses études historiques publiées jusqu'ici montrent que les forêts actuelles contiennent plus d'espèces feuillues intolérantes à l'ombre au détriment d'espèce de tolérance intermédiaire. La prolifération des coupes totales pourrait bien expliquer ce phénomène. Or, pour répondre aux objectifs d'aménagement écosystémique, il est possible que le MRNFQ ait à attribuer de plus grands volumes de ces deux essences, de manière à favoriser le retour à un état plus naturel de la forêt.
\\

Une augmentation de l'usage des coupes partielles (éclaircies, coupes progressives, etc.). Deux raisons principales portent à croire que l'avènement de l'aménagement écosystémique encouragera l'usage plus fréquent des coupes partielles pour approvisionner les usines. Premièrement, les coupes partielles favorisent la régénération des essences de tolérance intermédiaire à l'ombre. On a vu au point précédent que la régénération de ces essences représente un enjeu important dans la plupart des régions du Québec. Deuxièmement, les coupes partielles permettent de s'approvisionner en bois, même dans des contextes où la récolte est difficilement envisageable. Bien qu'elles puissent être affectées par diverses perturbations (feux, insectes, chablis, etc.), les forêts naturelles sont généralement dominées par des forêts mûres ou surannées (assez vieilles pour la récolte). Or, le maintien d'une part de forêts mûres dans le paysage représente un enjeu important, puisqu'à terme, le principe du rendement soutenu propose de rajeunir la forêt graduellement. En effet, en suivant ce principe, on fixe une période de révolution idéale à laquelle les arbres de seconde venue seront récoltés, ce qui ne laisse que très peu de place (ou pas du tout) pour les forêts mûres et surannées. Dans un contexte d'aménagement écosystémique, il est difficile de justifier la récolte forestière lorsque la quantité de ces forêts s'approche d'un seuil critique. La récolte par coupes partielles peut alors représenter un compromis intéressant puisque la forêt résiduelle maintient (ou regagnera plus rapidement) certains des attributs clés des vieilles forêts.


\subsection{Changements reliés à la mise en œuvre d'une sylviculture plus intensive}

En parallèle à la mise en œuvre de l'aménagement écosystémique, le nouveau régime forestier propose de réserver des superficies forestières à l'application d'une sylviculture intensive, vouée plus spécifiquement à la production de matière ligneuse. L'identification de zones de sylviculture intensive (ZSI) fait actuellement d'objet de réflexions et de discussions à travers la province. Évidemment, on ne mesurera qu'avec l'usage l'impact de la mise en œuvre de la stratégie de sylviculture intensive sur la qualité des approvisionnements. On peut toutefois entrevoir certains effets.\\

L'objectif poursuivi dans ces zones sera la production d'un volume utilisable dans des délais les plus courts possibles. On peut envisager qu'on aura notamment recours à la plantation et aux éclaircies afin d'y arriver. Ainsi, les espacements entre les arbres seront élargis par rapport à ce qui prévaut normalement dans la forêt naturelle.\\

Dans certains cas, on envisage avoir recours à certaines essences hybrides (peupliers et mélèzes hybrides) qui peuvent montrer des accroissements spectaculaires. On parle alors de ligniculture.\\

On a bien vu aux chapitres 7 et 8 les impacts potentiellement négatifs de tels choix. Il existe toutefois certaines opportunités qui feraient en sorte que la sylviculture intensive puisse mener à la production de bois de qualité. Ce serait le cas, par exemple, si on pratiquait l'élagage de manière plus régulière.\\

Il importe de réaliser que, contrairement à la mise en œuvre de l'aménagement écosystémique, les changements dans les pratiques forestières reliés à la sylviculture intensive n'aura pas un impact immédiat sur la qualité des approvisionnements.


%Référence
%MRNF 2009. Ressources et industries forestières portrait statistique édition 2009. Gouvernement du Québec, Ministère des Ressources naturelles et de la Faune [en ligne] http://www.mrnf.gouv.qc.ca/publications/forets/connaissances/stat_edition_complete/complete.pdf
